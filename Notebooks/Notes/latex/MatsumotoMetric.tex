\documentclass[11pt]{article}

    \usepackage[breakable]{tcolorbox}
    \usepackage{parskip} % Stop auto-indenting (to mimic markdown behaviour)
    
    \usepackage{iftex}
    \ifPDFTeX
    	\usepackage[T1]{fontenc}
    	\usepackage{mathpazo}
    \else
    	\usepackage{fontspec}
    \fi

    % Basic figure setup, for now with no caption control since it's done
    % automatically by Pandoc (which extracts ![](path) syntax from Markdown).
    \usepackage{graphicx}
    % Maintain compatibility with old templates. Remove in nbconvert 6.0
    \let\Oldincludegraphics\includegraphics
    % Ensure that by default, figures have no caption (until we provide a
    % proper Figure object with a Caption API and a way to capture that
    % in the conversion process - todo).
    \usepackage{caption}
    \DeclareCaptionFormat{nocaption}{}
    \captionsetup{format=nocaption,aboveskip=0pt,belowskip=0pt}

    \usepackage[Export]{adjustbox} % Used to constrain images to a maximum size
    \adjustboxset{max size={0.9\linewidth}{0.9\paperheight}}
    \usepackage{float}
    \floatplacement{figure}{H} % forces figures to be placed at the correct location
    \usepackage{xcolor} % Allow colors to be defined
    \usepackage{enumerate} % Needed for markdown enumerations to work
    \usepackage{geometry} % Used to adjust the document margins
    \usepackage{amsmath} % Equations
    \usepackage{amssymb} % Equations
    \usepackage{textcomp} % defines textquotesingle
    % Hack from http://tex.stackexchange.com/a/47451/13684:
    \AtBeginDocument{%
        \def\PYZsq{\textquotesingle}% Upright quotes in Pygmentized code
    }
    \usepackage{upquote} % Upright quotes for verbatim code
    \usepackage{eurosym} % defines \euro
    \usepackage[mathletters]{ucs} % Extended unicode (utf-8) support
    \usepackage{fancyvrb} % verbatim replacement that allows latex
    \usepackage{grffile} % extends the file name processing of package graphics 
                         % to support a larger range
    \makeatletter % fix for grffile with XeLaTeX
    \def\Gread@@xetex#1{%
      \IfFileExists{"\Gin@base".bb}%
      {\Gread@eps{\Gin@base.bb}}%
      {\Gread@@xetex@aux#1}%
    }
    \makeatother

    % The hyperref package gives us a pdf with properly built
    % internal navigation ('pdf bookmarks' for the table of contents,
    % internal cross-reference links, web links for URLs, etc.)
    \usepackage{hyperref}
    % The default LaTeX title has an obnoxious amount of whitespace. By default,
    % titling removes some of it. It also provides customization options.
    \usepackage{titling}
    \usepackage{longtable} % longtable support required by pandoc >1.10
    \usepackage{booktabs}  % table support for pandoc > 1.12.2
    \usepackage[inline]{enumitem} % IRkernel/repr support (it uses the enumerate* environment)
    \usepackage[normalem]{ulem} % ulem is needed to support strikethroughs (\sout)
                                % normalem makes italics be italics, not underlines
    \usepackage{mathrsfs}
    

    
    % Colors for the hyperref package
    \definecolor{urlcolor}{rgb}{0,.145,.698}
    \definecolor{linkcolor}{rgb}{.71,0.21,0.01}
    \definecolor{citecolor}{rgb}{.12,.54,.11}

    % ANSI colors
    \definecolor{ansi-black}{HTML}{3E424D}
    \definecolor{ansi-black-intense}{HTML}{282C36}
    \definecolor{ansi-red}{HTML}{E75C58}
    \definecolor{ansi-red-intense}{HTML}{B22B31}
    \definecolor{ansi-green}{HTML}{00A250}
    \definecolor{ansi-green-intense}{HTML}{007427}
    \definecolor{ansi-yellow}{HTML}{DDB62B}
    \definecolor{ansi-yellow-intense}{HTML}{B27D12}
    \definecolor{ansi-blue}{HTML}{208FFB}
    \definecolor{ansi-blue-intense}{HTML}{0065CA}
    \definecolor{ansi-magenta}{HTML}{D160C4}
    \definecolor{ansi-magenta-intense}{HTML}{A03196}
    \definecolor{ansi-cyan}{HTML}{60C6C8}
    \definecolor{ansi-cyan-intense}{HTML}{258F8F}
    \definecolor{ansi-white}{HTML}{C5C1B4}
    \definecolor{ansi-white-intense}{HTML}{A1A6B2}
    \definecolor{ansi-default-inverse-fg}{HTML}{FFFFFF}
    \definecolor{ansi-default-inverse-bg}{HTML}{000000}

    % commands and environments needed by pandoc snippets
    % extracted from the output of `pandoc -s`
    \providecommand{\tightlist}{%
      \setlength{\itemsep}{0pt}\setlength{\parskip}{0pt}}
    \DefineVerbatimEnvironment{Highlighting}{Verbatim}{commandchars=\\\{\}}
    % Add ',fontsize=\small' for more characters per line
    \newenvironment{Shaded}{}{}
    \newcommand{\KeywordTok}[1]{\textcolor[rgb]{0.00,0.44,0.13}{\textbf{{#1}}}}
    \newcommand{\DataTypeTok}[1]{\textcolor[rgb]{0.56,0.13,0.00}{{#1}}}
    \newcommand{\DecValTok}[1]{\textcolor[rgb]{0.25,0.63,0.44}{{#1}}}
    \newcommand{\BaseNTok}[1]{\textcolor[rgb]{0.25,0.63,0.44}{{#1}}}
    \newcommand{\FloatTok}[1]{\textcolor[rgb]{0.25,0.63,0.44}{{#1}}}
    \newcommand{\CharTok}[1]{\textcolor[rgb]{0.25,0.44,0.63}{{#1}}}
    \newcommand{\StringTok}[1]{\textcolor[rgb]{0.25,0.44,0.63}{{#1}}}
    \newcommand{\CommentTok}[1]{\textcolor[rgb]{0.38,0.63,0.69}{\textit{{#1}}}}
    \newcommand{\OtherTok}[1]{\textcolor[rgb]{0.00,0.44,0.13}{{#1}}}
    \newcommand{\AlertTok}[1]{\textcolor[rgb]{1.00,0.00,0.00}{\textbf{{#1}}}}
    \newcommand{\FunctionTok}[1]{\textcolor[rgb]{0.02,0.16,0.49}{{#1}}}
    \newcommand{\RegionMarkerTok}[1]{{#1}}
    \newcommand{\ErrorTok}[1]{\textcolor[rgb]{1.00,0.00,0.00}{\textbf{{#1}}}}
    \newcommand{\NormalTok}[1]{{#1}}
    
    % Additional commands for more recent versions of Pandoc
    \newcommand{\ConstantTok}[1]{\textcolor[rgb]{0.53,0.00,0.00}{{#1}}}
    \newcommand{\SpecialCharTok}[1]{\textcolor[rgb]{0.25,0.44,0.63}{{#1}}}
    \newcommand{\VerbatimStringTok}[1]{\textcolor[rgb]{0.25,0.44,0.63}{{#1}}}
    \newcommand{\SpecialStringTok}[1]{\textcolor[rgb]{0.73,0.40,0.53}{{#1}}}
    \newcommand{\ImportTok}[1]{{#1}}
    \newcommand{\DocumentationTok}[1]{\textcolor[rgb]{0.73,0.13,0.13}{\textit{{#1}}}}
    \newcommand{\AnnotationTok}[1]{\textcolor[rgb]{0.38,0.63,0.69}{\textbf{\textit{{#1}}}}}
    \newcommand{\CommentVarTok}[1]{\textcolor[rgb]{0.38,0.63,0.69}{\textbf{\textit{{#1}}}}}
    \newcommand{\VariableTok}[1]{\textcolor[rgb]{0.10,0.09,0.49}{{#1}}}
    \newcommand{\ControlFlowTok}[1]{\textcolor[rgb]{0.00,0.44,0.13}{\textbf{{#1}}}}
    \newcommand{\OperatorTok}[1]{\textcolor[rgb]{0.40,0.40,0.40}{{#1}}}
    \newcommand{\BuiltInTok}[1]{{#1}}
    \newcommand{\ExtensionTok}[1]{{#1}}
    \newcommand{\PreprocessorTok}[1]{\textcolor[rgb]{0.74,0.48,0.00}{{#1}}}
    \newcommand{\AttributeTok}[1]{\textcolor[rgb]{0.49,0.56,0.16}{{#1}}}
    \newcommand{\InformationTok}[1]{\textcolor[rgb]{0.38,0.63,0.69}{\textbf{\textit{{#1}}}}}
    \newcommand{\WarningTok}[1]{\textcolor[rgb]{0.38,0.63,0.69}{\textbf{\textit{{#1}}}}}
    
    
    % Define a nice break command that doesn't care if a line doesn't already
    % exist.
    \def\br{\hspace*{\fill} \\* }
    % Math Jax compatibility definitions
    \def\gt{>}
    \def\lt{<}
    \let\Oldtex\TeX
    \let\Oldlatex\LaTeX
    \renewcommand{\TeX}{\textrm{\Oldtex}}
    \renewcommand{\LaTeX}{\textrm{\Oldlatex}}
    % Document parameters
    % Document title
\title{Matsumoto's \textit{Slope of a mountain} metric}
    
    
    
    
    
% Pygments definitions
\makeatletter
\def\PY@reset{\let\PY@it=\relax \let\PY@bf=\relax%
    \let\PY@ul=\relax \let\PY@tc=\relax%
    \let\PY@bc=\relax \let\PY@ff=\relax}
\def\PY@tok#1{\csname PY@tok@#1\endcsname}
\def\PY@toks#1+{\ifx\relax#1\empty\else%
    \PY@tok{#1}\expandafter\PY@toks\fi}
\def\PY@do#1{\PY@bc{\PY@tc{\PY@ul{%
    \PY@it{\PY@bf{\PY@ff{#1}}}}}}}
\def\PY#1#2{\PY@reset\PY@toks#1+\relax+\PY@do{#2}}

\@namedef{PY@tok@w}{\def\PY@tc##1{\textcolor[rgb]{0.73,0.73,0.73}{##1}}}
\@namedef{PY@tok@c}{\let\PY@it=\textit\def\PY@tc##1{\textcolor[rgb]{0.25,0.50,0.50}{##1}}}
\@namedef{PY@tok@cp}{\def\PY@tc##1{\textcolor[rgb]{0.74,0.48,0.00}{##1}}}
\@namedef{PY@tok@k}{\let\PY@bf=\textbf\def\PY@tc##1{\textcolor[rgb]{0.00,0.50,0.00}{##1}}}
\@namedef{PY@tok@kp}{\def\PY@tc##1{\textcolor[rgb]{0.00,0.50,0.00}{##1}}}
\@namedef{PY@tok@kt}{\def\PY@tc##1{\textcolor[rgb]{0.69,0.00,0.25}{##1}}}
\@namedef{PY@tok@o}{\def\PY@tc##1{\textcolor[rgb]{0.40,0.40,0.40}{##1}}}
\@namedef{PY@tok@ow}{\let\PY@bf=\textbf\def\PY@tc##1{\textcolor[rgb]{0.67,0.13,1.00}{##1}}}
\@namedef{PY@tok@nb}{\def\PY@tc##1{\textcolor[rgb]{0.00,0.50,0.00}{##1}}}
\@namedef{PY@tok@nf}{\def\PY@tc##1{\textcolor[rgb]{0.00,0.00,1.00}{##1}}}
\@namedef{PY@tok@nc}{\let\PY@bf=\textbf\def\PY@tc##1{\textcolor[rgb]{0.00,0.00,1.00}{##1}}}
\@namedef{PY@tok@nn}{\let\PY@bf=\textbf\def\PY@tc##1{\textcolor[rgb]{0.00,0.00,1.00}{##1}}}
\@namedef{PY@tok@ne}{\let\PY@bf=\textbf\def\PY@tc##1{\textcolor[rgb]{0.82,0.25,0.23}{##1}}}
\@namedef{PY@tok@nv}{\def\PY@tc##1{\textcolor[rgb]{0.10,0.09,0.49}{##1}}}
\@namedef{PY@tok@no}{\def\PY@tc##1{\textcolor[rgb]{0.53,0.00,0.00}{##1}}}
\@namedef{PY@tok@nl}{\def\PY@tc##1{\textcolor[rgb]{0.63,0.63,0.00}{##1}}}
\@namedef{PY@tok@ni}{\let\PY@bf=\textbf\def\PY@tc##1{\textcolor[rgb]{0.60,0.60,0.60}{##1}}}
\@namedef{PY@tok@na}{\def\PY@tc##1{\textcolor[rgb]{0.49,0.56,0.16}{##1}}}
\@namedef{PY@tok@nt}{\let\PY@bf=\textbf\def\PY@tc##1{\textcolor[rgb]{0.00,0.50,0.00}{##1}}}
\@namedef{PY@tok@nd}{\def\PY@tc##1{\textcolor[rgb]{0.67,0.13,1.00}{##1}}}
\@namedef{PY@tok@s}{\def\PY@tc##1{\textcolor[rgb]{0.73,0.13,0.13}{##1}}}
\@namedef{PY@tok@sd}{\let\PY@it=\textit\def\PY@tc##1{\textcolor[rgb]{0.73,0.13,0.13}{##1}}}
\@namedef{PY@tok@si}{\let\PY@bf=\textbf\def\PY@tc##1{\textcolor[rgb]{0.73,0.40,0.53}{##1}}}
\@namedef{PY@tok@se}{\let\PY@bf=\textbf\def\PY@tc##1{\textcolor[rgb]{0.73,0.40,0.13}{##1}}}
\@namedef{PY@tok@sr}{\def\PY@tc##1{\textcolor[rgb]{0.73,0.40,0.53}{##1}}}
\@namedef{PY@tok@ss}{\def\PY@tc##1{\textcolor[rgb]{0.10,0.09,0.49}{##1}}}
\@namedef{PY@tok@sx}{\def\PY@tc##1{\textcolor[rgb]{0.00,0.50,0.00}{##1}}}
\@namedef{PY@tok@m}{\def\PY@tc##1{\textcolor[rgb]{0.40,0.40,0.40}{##1}}}
\@namedef{PY@tok@gh}{\let\PY@bf=\textbf\def\PY@tc##1{\textcolor[rgb]{0.00,0.00,0.50}{##1}}}
\@namedef{PY@tok@gu}{\let\PY@bf=\textbf\def\PY@tc##1{\textcolor[rgb]{0.50,0.00,0.50}{##1}}}
\@namedef{PY@tok@gd}{\def\PY@tc##1{\textcolor[rgb]{0.63,0.00,0.00}{##1}}}
\@namedef{PY@tok@gi}{\def\PY@tc##1{\textcolor[rgb]{0.00,0.63,0.00}{##1}}}
\@namedef{PY@tok@gr}{\def\PY@tc##1{\textcolor[rgb]{1.00,0.00,0.00}{##1}}}
\@namedef{PY@tok@ge}{\let\PY@it=\textit}
\@namedef{PY@tok@gs}{\let\PY@bf=\textbf}
\@namedef{PY@tok@gp}{\let\PY@bf=\textbf\def\PY@tc##1{\textcolor[rgb]{0.00,0.00,0.50}{##1}}}
\@namedef{PY@tok@go}{\def\PY@tc##1{\textcolor[rgb]{0.53,0.53,0.53}{##1}}}
\@namedef{PY@tok@gt}{\def\PY@tc##1{\textcolor[rgb]{0.00,0.27,0.87}{##1}}}
\@namedef{PY@tok@err}{\def\PY@bc##1{{\setlength{\fboxsep}{\string -\fboxrule}\fcolorbox[rgb]{1.00,0.00,0.00}{1,1,1}{\strut ##1}}}}
\@namedef{PY@tok@kc}{\let\PY@bf=\textbf\def\PY@tc##1{\textcolor[rgb]{0.00,0.50,0.00}{##1}}}
\@namedef{PY@tok@kd}{\let\PY@bf=\textbf\def\PY@tc##1{\textcolor[rgb]{0.00,0.50,0.00}{##1}}}
\@namedef{PY@tok@kn}{\let\PY@bf=\textbf\def\PY@tc##1{\textcolor[rgb]{0.00,0.50,0.00}{##1}}}
\@namedef{PY@tok@kr}{\let\PY@bf=\textbf\def\PY@tc##1{\textcolor[rgb]{0.00,0.50,0.00}{##1}}}
\@namedef{PY@tok@bp}{\def\PY@tc##1{\textcolor[rgb]{0.00,0.50,0.00}{##1}}}
\@namedef{PY@tok@fm}{\def\PY@tc##1{\textcolor[rgb]{0.00,0.00,1.00}{##1}}}
\@namedef{PY@tok@vc}{\def\PY@tc##1{\textcolor[rgb]{0.10,0.09,0.49}{##1}}}
\@namedef{PY@tok@vg}{\def\PY@tc##1{\textcolor[rgb]{0.10,0.09,0.49}{##1}}}
\@namedef{PY@tok@vi}{\def\PY@tc##1{\textcolor[rgb]{0.10,0.09,0.49}{##1}}}
\@namedef{PY@tok@vm}{\def\PY@tc##1{\textcolor[rgb]{0.10,0.09,0.49}{##1}}}
\@namedef{PY@tok@sa}{\def\PY@tc##1{\textcolor[rgb]{0.73,0.13,0.13}{##1}}}
\@namedef{PY@tok@sb}{\def\PY@tc##1{\textcolor[rgb]{0.73,0.13,0.13}{##1}}}
\@namedef{PY@tok@sc}{\def\PY@tc##1{\textcolor[rgb]{0.73,0.13,0.13}{##1}}}
\@namedef{PY@tok@dl}{\def\PY@tc##1{\textcolor[rgb]{0.73,0.13,0.13}{##1}}}
\@namedef{PY@tok@s2}{\def\PY@tc##1{\textcolor[rgb]{0.73,0.13,0.13}{##1}}}
\@namedef{PY@tok@sh}{\def\PY@tc##1{\textcolor[rgb]{0.73,0.13,0.13}{##1}}}
\@namedef{PY@tok@s1}{\def\PY@tc##1{\textcolor[rgb]{0.73,0.13,0.13}{##1}}}
\@namedef{PY@tok@mb}{\def\PY@tc##1{\textcolor[rgb]{0.40,0.40,0.40}{##1}}}
\@namedef{PY@tok@mf}{\def\PY@tc##1{\textcolor[rgb]{0.40,0.40,0.40}{##1}}}
\@namedef{PY@tok@mh}{\def\PY@tc##1{\textcolor[rgb]{0.40,0.40,0.40}{##1}}}
\@namedef{PY@tok@mi}{\def\PY@tc##1{\textcolor[rgb]{0.40,0.40,0.40}{##1}}}
\@namedef{PY@tok@il}{\def\PY@tc##1{\textcolor[rgb]{0.40,0.40,0.40}{##1}}}
\@namedef{PY@tok@mo}{\def\PY@tc##1{\textcolor[rgb]{0.40,0.40,0.40}{##1}}}
\@namedef{PY@tok@ch}{\let\PY@it=\textit\def\PY@tc##1{\textcolor[rgb]{0.25,0.50,0.50}{##1}}}
\@namedef{PY@tok@cm}{\let\PY@it=\textit\def\PY@tc##1{\textcolor[rgb]{0.25,0.50,0.50}{##1}}}
\@namedef{PY@tok@cpf}{\let\PY@it=\textit\def\PY@tc##1{\textcolor[rgb]{0.25,0.50,0.50}{##1}}}
\@namedef{PY@tok@c1}{\let\PY@it=\textit\def\PY@tc##1{\textcolor[rgb]{0.25,0.50,0.50}{##1}}}
\@namedef{PY@tok@cs}{\let\PY@it=\textit\def\PY@tc##1{\textcolor[rgb]{0.25,0.50,0.50}{##1}}}

\def\PYZbs{\char`\\}
\def\PYZus{\char`\_}
\def\PYZob{\char`\{}
\def\PYZcb{\char`\}}
\def\PYZca{\char`\^}
\def\PYZam{\char`\&}
\def\PYZlt{\char`\<}
\def\PYZgt{\char`\>}
\def\PYZsh{\char`\#}
\def\PYZpc{\char`\%}
\def\PYZdl{\char`\$}
\def\PYZhy{\char`\-}
\def\PYZsq{\char`\'}
\def\PYZdq{\char`\"}
\def\PYZti{\char`\~}
% for compatibility with earlier versions
\def\PYZat{@}
\def\PYZlb{[}
\def\PYZrb{]}
\makeatother


    % For linebreaks inside Verbatim environment from package fancyvrb. 
    \makeatletter
        \newbox\Wrappedcontinuationbox 
        \newbox\Wrappedvisiblespacebox 
        \newcommand*\Wrappedvisiblespace {\textcolor{red}{\textvisiblespace}} 
        \newcommand*\Wrappedcontinuationsymbol {\textcolor{red}{\llap{\tiny$\m@th\hookrightarrow$}}} 
        \newcommand*\Wrappedcontinuationindent {3ex } 
        \newcommand*\Wrappedafterbreak {\kern\Wrappedcontinuationindent\copy\Wrappedcontinuationbox} 
        % Take advantage of the already applied Pygments mark-up to insert 
        % potential linebreaks for TeX processing. 
        %        {, <, #, %, $, ' and ": go to next line. 
        %        _, }, ^, &, >, - and ~: stay at end of broken line. 
        % Use of \textquotesingle for straight quote. 
        \newcommand*\Wrappedbreaksatspecials {% 
            \def\PYGZus{\discretionary{\char`\_}{\Wrappedafterbreak}{\char`\_}}% 
            \def\PYGZob{\discretionary{}{\Wrappedafterbreak\char`\{}{\char`\{}}% 
            \def\PYGZcb{\discretionary{\char`\}}{\Wrappedafterbreak}{\char`\}}}% 
            \def\PYGZca{\discretionary{\char`\^}{\Wrappedafterbreak}{\char`\^}}% 
            \def\PYGZam{\discretionary{\char`\&}{\Wrappedafterbreak}{\char`\&}}% 
            \def\PYGZlt{\discretionary{}{\Wrappedafterbreak\char`\<}{\char`\<}}% 
            \def\PYGZgt{\discretionary{\char`\>}{\Wrappedafterbreak}{\char`\>}}% 
            \def\PYGZsh{\discretionary{}{\Wrappedafterbreak\char`\#}{\char`\#}}% 
            \def\PYGZpc{\discretionary{}{\Wrappedafterbreak\char`\%}{\char`\%}}% 
            \def\PYGZdl{\discretionary{}{\Wrappedafterbreak\char`\$}{\char`\$}}% 
            \def\PYGZhy{\discretionary{\char`\-}{\Wrappedafterbreak}{\char`\-}}% 
            \def\PYGZsq{\discretionary{}{\Wrappedafterbreak\textquotesingle}{\textquotesingle}}% 
            \def\PYGZdq{\discretionary{}{\Wrappedafterbreak\char`\"}{\char`\"}}% 
            \def\PYGZti{\discretionary{\char`\~}{\Wrappedafterbreak}{\char`\~}}% 
        } 
        % Some characters . , ; ? ! / are not pygmentized. 
        % This macro makes them "active" and they will insert potential linebreaks 
        \newcommand*\Wrappedbreaksatpunct {% 
            \lccode`\~`\.\lowercase{\def~}{\discretionary{\hbox{\char`\.}}{\Wrappedafterbreak}{\hbox{\char`\.}}}% 
            \lccode`\~`\,\lowercase{\def~}{\discretionary{\hbox{\char`\,}}{\Wrappedafterbreak}{\hbox{\char`\,}}}% 
            \lccode`\~`\;\lowercase{\def~}{\discretionary{\hbox{\char`\;}}{\Wrappedafterbreak}{\hbox{\char`\;}}}% 
            \lccode`\~`\:\lowercase{\def~}{\discretionary{\hbox{\char`\:}}{\Wrappedafterbreak}{\hbox{\char`\:}}}% 
            \lccode`\~`\?\lowercase{\def~}{\discretionary{\hbox{\char`\?}}{\Wrappedafterbreak}{\hbox{\char`\?}}}% 
            \lccode`\~`\!\lowercase{\def~}{\discretionary{\hbox{\char`\!}}{\Wrappedafterbreak}{\hbox{\char`\!}}}% 
            \lccode`\~`\/\lowercase{\def~}{\discretionary{\hbox{\char`\/}}{\Wrappedafterbreak}{\hbox{\char`\/}}}% 
            \catcode`\.\active
            \catcode`\,\active 
            \catcode`\;\active
            \catcode`\:\active
            \catcode`\?\active
            \catcode`\!\active
            \catcode`\/\active 
            \lccode`\~`\~ 	
        }
    \makeatother

    \let\OriginalVerbatim=\Verbatim
    \makeatletter
    \renewcommand{\Verbatim}[1][1]{%
        %\parskip\z@skip
        \sbox\Wrappedcontinuationbox {\Wrappedcontinuationsymbol}%
        \sbox\Wrappedvisiblespacebox {\FV@SetupFont\Wrappedvisiblespace}%
        \def\FancyVerbFormatLine ##1{\hsize\linewidth
            \vtop{\raggedright\hyphenpenalty\z@\exhyphenpenalty\z@
                \doublehyphendemerits\z@\finalhyphendemerits\z@
                \strut ##1\strut}%
        }%
        % If the linebreak is at a space, the latter will be displayed as visible
        % space at end of first line, and a continuation symbol starts next line.
        % Stretch/shrink are however usually zero for typewriter font.
        \def\FV@Space {%
            \nobreak\hskip\z@ plus\fontdimen3\font minus\fontdimen4\font
            \discretionary{\copy\Wrappedvisiblespacebox}{\Wrappedafterbreak}
            {\kern\fontdimen2\font}%
        }%
        
        % Allow breaks at special characters using \PYG... macros.
        \Wrappedbreaksatspecials
        % Breaks at punctuation characters . , ; ? ! and / need catcode=\active 	
        \OriginalVerbatim[#1,codes*=\Wrappedbreaksatpunct]%
    }
    \makeatother

    % Exact colors from NB
    \definecolor{incolor}{HTML}{303F9F}
    \definecolor{outcolor}{HTML}{D84315}
    \definecolor{cellborder}{HTML}{CFCFCF}
    \definecolor{cellbackground}{HTML}{F7F7F7}
    
    % prompt
    \makeatletter
    \newcommand{\boxspacing}{\kern\kvtcb@left@rule\kern\kvtcb@boxsep}
    \makeatother
    \newcommand{\prompt}[4]{
        \ttfamily\llap{{\color{#2}[#3]:\hspace{3pt}#4}}\vspace{-\baselineskip}
    }
    

    
    % Prevent overflowing lines due to hard-to-break entities
    \sloppy 
    % Setup hyperref package
    \hypersetup{
      breaklinks=true,  % so long urls are correctly broken across lines
      colorlinks=true,
      urlcolor=urlcolor,
      linkcolor=linkcolor,
      citecolor=citecolor,
      }
    % Slightly bigger margins than the latex defaults
    
    \geometry{verbose,tmargin=1in,bmargin=1in,lmargin=1in,rmargin=1in}
    
    

\pagecolor{white}
\author{Colin P. Stark}
\begin{document}
    
    \maketitle
    
    

    
    \hypertarget{summary}{%
\section*{Summary}\label{summary}}


    In the first of three 2019 lectures on Finsler spacetime, Javaloyes
summarizes the Matsumoto metric like this:

 
            
    
    \begin{center}
    \adjustimage{max size={0.9\linewidth}{0.9\paperheight}}{MatsumotoMetric_files/MatsumotoMetric_3_0.png}
    \end{center}
    { \hspace*{\fill} \\}
    

    Matsumoto's idea -- which was inspired by a question raised by Paul
Finsler himself -- is to measure how far a mountain hiker will walk in a
given direction in unit time \emph{if they maintain the same effort
regardless of the local slope}.

The physics boils down to this:

\begin{itemize}
\tightlist
\item
  choose a map direction \(\phi\) for the hike
\item
  specify a step interval \(\Delta{t} \ll {}\) unit time
\item
  take a step in the chosen direction with an fixed impulse
\item
  terminate the step with another fixed impulse
\item
  choose this impulse pair so that \emph{on a horizontal surface} a
  constant chosen speed \(c\) is maintained
\item
  on a sloping surface, the speed will be faster or slower than \(c\)
  depending on the tilt
\item
  include in the termination impulse a lateral component to compensate
  for any ``off-axis'' acceleration and to ensure the path continues in
  the chosen direction
\item
  keep stepping so that a steady pace is established
\item
  measure the distance traveled in unit time (ignoring the vagaries of
  the first few steps)
\end{itemize}

Horizontal walking speed in the chosen direction is set by two things:

\begin{enumerate}
\def\labelenumi{\arabic{enumi}.}
\tightlist
\item
  the step-wise accelerations and decelerations induced by the walking
  impulses
\item
  the step-wise accelerations induced by the component of gravity
  resolved in the walking direction
\end{enumerate}

    \hypertarget{derivation}{%
\section*{Derivation}\label{derivation}}

    \hypertarget{original-version-hiker-on-a-mountain-slope}{%
\subsection*{Original version: Hiker on a mountain
slope}\label{original-version-hiker-on-a-mountain-slope}}

    Matsumoto's original paper invokes a hiker who somehow achieves an
orientation-dependent ``terminal velocity'' after a few steps. The
physics of this seem a bit vague to me, so in the next section I have
adapted it into a more carefully crafted form. First, though, here is
the Bao \& Robles (2004) version of Matsumoto (1989)

    \hypertarget{preliminaries}{%
\subsubsection*{Preliminaries}\label{preliminaries}}

 
            
    
    \begin{center}
    \adjustimage{max size={0.9\linewidth}{0.9\paperheight}}{MatsumotoMetric_files/MatsumotoMetric_9_0.png}
    \end{center}
    { \hspace*{\fill} \\}
    

    A topographic surface \(S\) in 3D space is defined by a function
\(z=f(x,y)\) and described parametrically by \((x,y,f(x,y))\) {[}see
Matsumoto, 1989, p.22{]}.

The tangent plane at (\(x\),\(y\)) has local basis vectors \((1,0,f_x)\)
and \((0,1,f_y)\) and these are respectively abusively denoted by
\(\partial_x\) and \(\partial_y\) (abusive, because they not simple
derivatives). The dual basis is denoted by \(dx, dy\).

We can define a covariant metric tensor \(g\) (Bao \& Robles denote it
as \(h\)) for this surface using the mutual Euclidean (3D) lengths of
the vectors \((1,0,f_x)\) and \((0,1,f_y)\) and the tensor products of
the dual basis elements:

\begin{align}
    g
%    :=&  \, \partial_x \otimes \partial_x + \partial_x \otimes \partial_y + \partial_y \otimes \partial_x + \partial_y \otimes \partial_y \\
%    =&
%    \begin{bmatrix}
%    \partial_x & \partial_y
%    \end{bmatrix}
%    \otimes 
%    \begin{bmatrix}
%   \partial_x &  \partial_y
%    \end{bmatrix} 
%    \\
%    =& \,
%    \begin{bmatrix}
%    (1,0,f_x) & (0,1,f_y)
%    \end{bmatrix}
%    \otimes
%    \begin{bmatrix}
%        (1,0,f_x) & (0,1,f_y)
%    \end{bmatrix} \\
%    =& \, 
%    \begin{bmatrix}
%    [(1+f_x^2)\, dx \otimes dx & f_x f_y \, dx \otimes dy] & [f_y f_x \, dy \otimes dx & \, (1+f_y^2) \, dy \otimes dy] 
%    \end{bmatrix}
%    \\
    =& \, 
      \left((1,0,f_x)\cdot(1,0,f_x)\right) \, dx \otimes dx 
    + \left((1,0,f_x)\cdot(0,1,f_y)\right) \, (dx \otimes dy + dy \otimes dx) 
    + \left((0,1,f_y)\cdot(0,1,f_y)\right) \, dy \otimes dy
    \\
    =& \, (1+f_x^2)\, dx \otimes dx + f_x f_y \, (dx \otimes dy + dy \otimes dx) + (1+f_y^2)\, dy \otimes dy
    \\
    =& 
    \begin{bmatrix}
        (1+f_x^2) & f_x f_y  \\
        f_x f_y  & (1+f_y^2)
    \end{bmatrix}
\end{align}

    The metric tensor provides a means of measuring the length of a tangent
vector \(Y := u\partial_x + v\partial_y\) on \(S\):

\begin{align}
    \| Y \|_{g}^2
    &= g(Y,Y) = g_{ij} Y^i Y^j\\
    &= 
    \begin{bmatrix}
    u & v
    \end{bmatrix}
    \begin{bmatrix}
        (1+f_x^2) & f_x f_y  \\
        f_y f_x  & (1+f_y^2)
    \end{bmatrix}
    \begin{bmatrix}
    u \\ v
    \end{bmatrix} 
    \\
    &=
    \begin{bmatrix}
    u & v
    \end{bmatrix}
    \begin{bmatrix}
    u(1+f_x^2) + v \,f_x f_y  \\ 
    u \,f_y f_x   + v (1+f_y^2)
    \end{bmatrix} \\
    &= u^2(1+f_x^2) + v uf_x f_y  + u v f_y f_x   + v^2(1+f_y^2)  \\
    &= u^2 + v^2 + u^2 f_x^2  + 2 u v f_x  f_y  + v^2 f_y^2 \\
    &= u^2 + v^2 + (u f_x  +  vf_y)^2
\end{align}

    The dual (contravariant) metric tensor is the inverse of \(g\)
\begin{equation}
    g^{-1}
    =
    \begin{bmatrix}
        (1+f_x^2) & f_x f_y  \\
        f_y f_x  & (1+f_y^2)
    \end{bmatrix}^{-1}
%    =
%    \begin{bmatrix}
%        (1+f_y^2) & -f_x f_y  \\
%        -f_y f_x  & (1+f_x^2)
%    \end{bmatrix}\bigg/((1+f_x^2)(1+f_y^2)-f_x^2 f_y^2)
    =
    \dfrac{1}{1+f_x^2+f_y^2}
    \begin{bmatrix}
        {1+f_y^2}  & -{f_x f_y}  \\
        -{f_y f_x} & {1+f_x^2}
    \end{bmatrix}
\end{equation}

    We can also map the covector differential \(df\) to its dual
contravariant vector using \(g^{-1}\):

\begin{align}
    (df)^{\sharp}  
    & = g^{-1} df\\
    &= 
    \dfrac{
        1
    }{1+f_x^2+f_y^2}
    \begin{bmatrix}
        {1+f_y^2}  & -{f_x f_y}  \\
        -{f_y f_x} & {1+f_x^2}
    \end{bmatrix}
    \begin{bmatrix}
        f_x & f_y
    \end{bmatrix}^T
    \\
    &= 
    \dfrac{1}{1+f_x^2+f_y^2}
    \begin{bmatrix}
    (1+f_y^2)f_x  - {f_x f_y} f_y \\
    -{f_y f_x}f_x + (1+f_x^2)f_y
    \end{bmatrix}
    \\
    &= 
    \dfrac{1}{1+f_x^2+f_y^2}
    \begin{bmatrix}
    f_x \\ f_y
    \end{bmatrix}
    \\
    &= 
    \dfrac{f_x \partial_x + f_y \partial_y}{1+f_x^2+f_y^2}
\end{align}

The length of this vector is

\begin{align}
    \|(df)^{\sharp}\|_g^2
    & = g\left((df)^{\sharp}, (df)^{\sharp}\right) \\
    & = g g^{-1} df\, g^{-1} df \\
    & = df\, g^{-1} \,df \\
    &= 
    \dfrac{1}{1+f_x^2+f_y^2}
    \begin{bmatrix}
    f_x & f_y
    \end{bmatrix}
    \begin{bmatrix}
        (1+f_x^2) & -f_x f_y  \\
        -f_y f_x  & (1+f_y^2)
    \end{bmatrix}
    \begin{bmatrix}
    f_x \\ f_y
    \end{bmatrix} 
    \\
    &=
    \dfrac{1}{1+f_x^2+f_y^2}
    \begin{bmatrix}
    f_x & f_y
    \end{bmatrix}
    \begin{bmatrix}
    f_x(1+f_x^2) - f_y \,f_x f_y  \\ 
    -f_x \,f_y f_x + f_y (1+f_y^2)
    \end{bmatrix} 
    \\
    &=
    \dfrac{1}{1+f_x^2+f_y^2}
    \begin{bmatrix}
    f_x & f_y
    \end{bmatrix}
    \begin{bmatrix}
    f_x  \\ 
    f_y
    \end{bmatrix} 
    \\
    &=
    \dfrac{f_x^2+f_y^2}{1+f_x^2+f_y^2}
\end{align}

    \hypertarget{model}{%
\subsubsection*{Model}\label{model}}

 
            
    
    \begin{center}
    \adjustimage{max size={0.9\linewidth}{0.9\paperheight}}{MatsumotoMetric_files/MatsumotoMetric_15_0.png}
    \end{center}
    { \hspace*{\fill} \\}
    

    \hypertarget{modified-version-robot-on-a-mountain-slope}{%
\subsection*{Modified version: Robot on a mountain
slope}\label{modified-version-robot-on-a-mountain-slope}}

    Imagine a bipedal robot walking across a smooth plane inclined towards
the south at an angle \(\phi\) from the horizontal. Between one step and
the next, the robot is surface-parallel accelerated or decelerated
depending on the orientation \(\theta\) of its motion relative to south.
As a step ends, the robot applies just the right impulse through its
feet that it:

\begin{enumerate}
\def\labelenumi{\arabic{enumi}.}
\tightlist
\item
  compensates for any step-orthogonal motion making it veer away from
  the intended direction of motion;
\item
  recovers the step-wise, surface-parallel speed \(c\) as it begins the
  next step.
\end{enumerate}

In this way, it maintains \emph{on average} a constant step-wise,
surface-parallel speed (that may be greater or less than \(c\)) in the
chosen direction of motion.

The question is, how far will it walk in a given direction \(\theta\) in
a fixed time averaged over many steps?

The incremental distance \(\Delta{r}\) traversed between each step
interval \(\Delta{t}\) will be the result of (a) the speed \(v\)
asserted when each step is taken, plus (b) the speed boost (or loss)
from the component of acceleration due to gravity \(g_{\mathrm{acc}}\)
acting along the generally non-horizontal path.

\begin{equation}
    \Delta{r} = c\Delta{t} 
        + \tfrac{1}{2} g_{\mathrm{acc}} \Delta{t}^2 \sin\phi\,\cos\theta 
\end{equation}

\begin{equation}
    \dfrac{\Delta{r}}{\Delta{t}}
        = c + \tfrac{1}{2} g_{\mathrm{acc}} \Delta{t} \sin\phi\,\cos\theta 
\end{equation}

So the normalized speed is:

\begin{equation}
    \tilde{c}
        = 1 + a \cos\theta 
\end{equation}

where

\begin{equation}
    a = \dfrac{g_{\mathrm{acc}}\Delta{t}}{2c} \sin\phi
\end{equation}

As the time interval \(\Delta{t}\) is reduced (to make motion smoother),
\(c\) is reduced to ensure \(a\) remains constant.

As far as the robot is concerned, assuming it can only sense enough to
maintain surface-parallel \(v\) at each step, distance appears to vary
with orientation. In other words, its distance metric is strongly
anisotropic and Finsler: it has a form that cannot be expressed in a
simple inner product that applies a metric tensor dependent only on
position: if it were, a Riemannian metric would apply. The anisotropy of
this metric is encapsulated in its speed-vs-orientation indicatrix,
which takes the interesting form of a \emph{limaçon}.

Note that, blithely ignoring units, Bao \& Robles assume
\(c = g_{\mathrm{acc}}/2\) towards the end of their explanation, such
that

\begin{equation}
    a = \Delta{t} \sin\phi
\end{equation}

Presumably they also assume \(\Delta{t}=1\), and since their
\((f_x, f_y)\) is here \((0,\tan\phi)\),

\begin{equation}
    f_y^2 =  \dfrac{1}{1-a^2} - 1 =  \dfrac{a^2}{1-a^2}
\end{equation}

or

\begin{equation}
    a^2 =  1-\dfrac{1}{1+\tan^2\phi}  =  \dfrac{\tan^2\phi}{1+\tan^2\phi} =  \dfrac{f_y^2}{1+f_y^2}
\end{equation}

If \(f_x=0\), \(f_y^2<\tfrac{1}{3}\), meaning the surface tilt must be
less than \(\phi < \pi/6\), then

\begin{equation}
    a < \sqrt{\dfrac{1/3}{1+1/3}} = \dfrac{1}{2}
\end{equation}

which means that most of the interesting limaçon-shaped metrics
\(a\geq 1/2\) are not strongly convex, and thus are not Finsler metrics
\emph{sensu stricto}.

    Let's simplify the Bao \& Robles version of the Matsumoto metric using
again the assumption that the surface is planar and tilted only to the
south at an angle \(f_y = \tan\phi\). Then

\begin{equation}
    |Y|^2 = u^2 + v^2 + v^2 f_y^2 = u^2 + v^2 (1+f_y^2) 
\end{equation} and

\begin{equation}
    |(df)^\#|^2 = \dfrac{f_y^2}{1+f_y^2}
\end{equation}

which makes the metric function

\begin{equation}
    F = \dfrac{|Y|^2}{|Y|-v f_y} = \dfrac{u^2 + v^2 (1+f_y^2) }{\sqrt{u^2 + v^2 (1+f_y^2)}-v f_y}
\end{equation}

    \begin{equation}
    F = \dfrac{|Y|^2}{|Y|-v f_y} = \dfrac{u^2 + v^2 (1+f_y^2) }{\sqrt{u^2 + v^2 (1+f_y^2)}-v f_y}
\end{equation}

    \hypertarget{plots-of-the-indicatrix-and-figuratrix}{%
\subsection*{Plots of the indicatrix and
figuratrix}\label{plots-of-the-indicatrix-and-figuratrix}}





    \begin{center}
    \adjustimage{max size={0.9\linewidth}{0.9\paperheight}}{MatsumotoMetric_files/MatsumotoMetric_24_0.png}
    \end{center}
    { \hspace*{\fill} \\}
    

    \begin{center}
    \adjustimage{max size={0.9\linewidth}{0.9\paperheight}}{MatsumotoMetric_files/MatsumotoMetric_25_0.png}
    \end{center}
    { \hspace*{\fill} \\}
    
    \hypertarget{references}{%
\section*{References}\label{references}}

    \href{http://doi.org/10.1007/978-94-009-5329-1}{Asanov, G. S. (1985).
Finsler Geometry, Relativity and Gauge Theories. Springer Science \&
Business Media.}

\href{http://library.msri.org/books/Book50/contents.html}{Bao, D., \&
Robles, C. (2004). Ricci and Flag Curvatures in Finsler Geometry. In D.
Bao, R. L. Bryant, S.-S. Chern, \& Z. Shen (Eds.), Riemann-Finsler
Geometry (Vol. 50, pp.~197--259).}

\href{http://doi.org/10.1007/978-3-642-18855-8}{Gallot, S., Hulin, D.,
\& Lafontaine, J. (2004). Riemannian geometry (3rd ed., pp.~1--337).
Springer.}

\href{http://doi.org/10.1063/5.0065944}{Hohmann, M., Pfeifer, C., \&
Voicu, N. (2022). Mathematical foundations for field theories on Finsler
spacetimes. Journal of Mathematical Physics, 63(3), 032503--48.}

\href{https://www.icmat.es/congresos/2019/IFWGP/program.php}{Javaloyes,
M. Á. (2019). Finsler Geometry: Riemannian foundations and relativistic
applications. Lecture 1: Definitions, examples and basic properties
(pp.~1--122). Presented at the XXVIII International Fall Workshop on
Geometry and Physics ICMAT.}

\href{http://projecteuclid.org/download/pdf_1/euclid.kjm/1250520303}{Matsumoto,
M. (1989). A slope of a mountain is a Finsler surface with respect to a
time measure. Journal of Mathematics of Kyoto University, 29(1),
17--25.}

    \hypertarget{appendix}{%
\section*{Appendix}\label{appendix}}

    For reference, here is the conservative definition of a Finsler metric
from Bao \& Robles {[}2004{]}, pp.199-200. It adopts the requirement
that the fundamental tensor \(g_{ij}\) be positive definite.

 
            
    
    \begin{center}
    \adjustimage{max size={0.9\linewidth}{0.9\paperheight}}{MatsumotoMetric_files/MatsumotoMetric_30_0.png}
    \end{center}
    { \hspace*{\fill} \\}
    

    However, Asanov {[}1985{]} is more relaxed, noting that by insisting on
positive-definiteness we throw out any GR applications, since
pseudo-Riemannian manifolds are not positive definite:

 
            
    
    \begin{center}
    \adjustimage{max size={0.9\linewidth}{0.9\paperheight}}{MatsumotoMetric_files/MatsumotoMetric_32_0.png}
    \end{center}
    { \hspace*{\fill} \\}
    

    Gallot et al {[}2004{]} discuss some of the consequences of allowing for
a mixed signature (metric tensor with negative eigenvalues) in the
context of Riemannian manifolds (they don't touch on Finsler spaces):

 
            
    
    \begin{center}
    \adjustimage{max size={0.9\linewidth}{0.9\paperheight}}{MatsumotoMetric_files/MatsumotoMetric_34_0.png}
    \end{center}
    { \hspace*{\fill} \\}
    

    At the risk of terminal obfuscation, here is a very recent paper by
Hohmann et al {[}2022{]} and their prescription of a Finsler spacetime:
note how, echoing Asanov, they require only that the fundamental tensor
be non-degenerate. I don't understand how they deal with light-like
(null) curves, but that would seem to be precisely where the determinant
of \(g_{ij}\) is zero\ldots{}

 
            
    
    \begin{center}
    \adjustimage{max size={0.9\linewidth}{0.9\paperheight}}{MatsumotoMetric_files/MatsumotoMetric_36_0.png}
    \end{center}
    { \hspace*{\fill} \\}
    



    % Add a bibliography block to the postdoc
    
    
    
\end{document}
